\documentclass[a4paper]{article}

\usepackage[portuguese]{babel}
\usepackage[utf8]{inputenc}
\usepackage[T1]{fontenc}

\newcommand{\documentTitle}{Genetic Algorithms - Brachistochrone Curve} %Macro definition
\newcommand{\documentAuthors}{João Rafael (2008111876, jprafael@student.dei.uc.pt) \and José Ribeiro (2008112181, jbaia@student.dei.uc.pt)} %Macro definition

\title{\documentTitle}
\author{\documentAuthors{}}

\usepackage{hyperref}
\hypersetup{
	pdftitle = \documentTitle
	,pdfauthor = \documentAuthors
	,pdfsubject = {Introduction to Artificial Inteligence Project \#2 Report}
	,pdfkeywords = {Artificial Inteligence Project} {Genetic Algorithms} {Brachistochrone Curve}
	,pdfborder = {0 0 0}
}

\usepackage{subfig}
\usepackage{amsmath}
\usepackage{wrapfig}
\usepackage{array}
\usepackage{anysize}
\usepackage{lscape}
\usepackage[pdftex]{graphicx}

\marginsize{3.5cm}{3.5cm}{3cm}{3cm}

\makeatletter

\begin{document}
\renewcommand{\figurename}{Figure}
\maketitle
\cleardoublepage

\tableofcontents
\cleardoublepage

\setlength{\parindent}{1cm}
\setlength{\parskip}{0.3cm}

\section{Introduction}
\indent \indent ...

...

\cleardoublepage
\section{Implementação}
\indent \indent ...

\cleardoublepage

\subsection{...}
\indent \indent ...

\subsection{...}
\indent \indent ...

\subsubsection{...}
\indent \indent ...

\subsubsection{...}
\indent \indent ...

...

\cleardoublepage
\subsection{...}
\indent \indent ...

...

\subsection{...}
\indent \indent ...

...

\subsection{...}
\indent \indent ...

\cleardoublepage
\section{Experimentação}
\indent \indent ...

\cleardoublepage

\subsection{...}
\indent \indent ...

\cleardoublepage
\section{Validação}
\indent \indent Para validar o comportamento do Agente, cada combinação de parâmetros foi executada 30 vezes com seeds diferentes;
tal garante que a natureza estocástica do algoritmo fica evidenciada ao garantir que os testes são, de facto, diferentes.

Após a conclusão dos testes, foi calculado o desvio-padrão entre as 30 simulações,
para averiguar se os resultados obtidos não representavam apenas um caso de sorte.

Tal como se pode verificar na última coluna da tabela de resultados (Secção \ref{tab:results}), o desvio-padrão entre as 30 simulações para cada combinação de parâmetros
é bastante reduzido, o que comprova a sua consistência e robustez.

\cleardoublepage

\subsection{...}
\indent \indent ...

\cleardoublepage
\section{Análise de resultados}
\indent \indent ...

\cleardoublepage

\subsection{...}
\indent \indent ...

\cleardoublepage
\section{Conclusão}
\indent \indent ...

\cleardoublepage

\subsection{...}
\indent \indent ...

\eject \pdfpagewidth=594.0mm \pdfpageheight=420.0mm

\cleardoublepage
\section{Anexos}
\subsection{Tabela de Resultados}
\begin{center}
	\begin{tabular}{ | c | c | c | c | c | c | c | c | c | c | c | c | c | c | c | c | c | }
		\hline
		\textbf{Points set}	&	\textbf{Representation}	&	\textbf{Selection Type}	&	\textbf{No. pts.}	&	\textbf{Pop. size}	&	\textbf{Elitism  (\%)}	&	\textbf{Crossover Prob. (\%)}	&	\textbf{No. pts. Crossover}	&	\textbf{Prob. Mutation (\%)}	&	\textbf{Best (20)}	&	\textbf{Best (100)}	&	\textbf{Best (1000)}	&	\textbf{Best (2000)}	&	\textbf{Average (2000)}	&	\textbf{Worst (2000)}	&	\textbf{Std. Dev. (2000, Pop.)}	&	\textbf{Std. Dev. x100 (2000, 30 runs)} \\
		\hline
		\hline
		1	&	Dynamic	&	Tournment	&	30	&	100	&	10	&	10	&	3	&	10	&	3.0031230	&	3.0031230	&	3.0031230	&	3.0031230	&	3.0031230	&	3.0031230	&	0.0031230	&	0.0031230 \\
		\hline
		2	&	Dynamic	&	Tournment	&	15	&	50	&	10	&	10	&	3	&	10	&	2.0023430	&	5.2323430	&	3.2331230	&	3.0031230	&	3.0031230	&	3.0031230	&	0.0031230	&	0.0031230 \\
		\hline
	\end{tabular}
	\label{tab:results}
\end{center}


\end{document}
