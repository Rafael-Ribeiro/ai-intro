\documentclass[a4paper]{article}

\usepackage[portuguese]{babel}
\usepackage[utf8]{inputenc}
\usepackage[T1]{fontenc}

\newcommand{\documentTitle}{Genetic Algorithms - Brachistochrone Curve} %Macro definition
\newcommand{\documentAuthors}{João Rafael (2008111876, jprafael@student.dei.uc.pt) \and José Ribeiro (2008112181, jbaia@student.dei.uc.pt)} %Macro definition

\title{\documentTitle}
\author{\documentAuthors{}}

\usepackage{hyperref}
\hypersetup{
	pdftitle = \documentTitle
	,pdfauthor = \documentAuthors
	,pdfsubject = {Introduction to Artificial Inteligence Project \#2 Report}
	,pdfkeywords = {Artificial Inteligence Project} {Genetic Algorithms} {Brachistochrone Curve}
	,pdfborder = {0 0 0}
}

\usepackage{subfig}
\usepackage{amsmath}
\usepackage{wrapfig}
\usepackage{array}
\usepackage{anysize}
\usepackage{lscape}
\usepackage[pdftex]{graphicx}

\marginsize{3.5cm}{3.5cm}{3cm}{3cm}

\makeatletter

\begin{document}
\renewcommand{\figurename}{Figure}
\maketitle
\cleardoublepage

\tableofcontents
\cleardoublepage

\setlength{\parindent}{1cm}
\setlength{\parskip}{0.3cm}

\section{Introduction}
\indent \indent Este projecto está inserido no âmbito da disciplina de Introdução à Inteligência Artificial,
mais concretamente no seguimento do primeiro projecto, uma vez que implementamos outro tipo de Agentes: Agentes Adaptativos.

Ao contrário dos agentes já estudados (Reactivos), estes baseiam-se fundamentalmente em conceitos da Biologia, nomeadamente a teoria da Selecção Natural de Darwin aplicada à genética.
Estes agentes representam populações, que ao longo do tempo (iterações da aplicação), evoluem ao sofrer mutações e recombinações entre individos (e os seus genes) e posterior selecção dos mais aptos.
Desta forma pretende-se que a aptidão da população melhore, convergindo para o óptimo global.

\subsection{Brachistochrone curve}
\indent \indent O problema da curva braquistócrona é um clásico da disciplina de cálculo:

\emph{Tendo dois pontos distintos, A e B, o objectivo é conhecer a trajectória que minimiza o tempo que um ponto material demora a deslocar-se entre eles, quando sujeito apenas à força da gravidade (com atrito desprezável).}

\indent Este problema apenas é válido quando se consideram pares de pontos com a altura de B inferior à de A, pois em caso contrário o corpo não consegue efectuar o percurso.

Leibniz, L'Hospital, Newton, e os irmãos Bernoulli apresentaram soluções analíticas.
No entanto, o estudo deste problema segundo o paradigma de agentes adaptativos é interessante pois permite calcular uma aproximação da curva
não necessitando de ferramentas matemáticas complexas.

\indent Para cada instância do problema, um individuo representa uma trajectória possível. Cada um deles é caracterizado por um conjunto de genes
que dependendo do modelo utilizado, representam os varios factores que contribuem para a forma da trajectória. Estes individuos são avaliados segundo uma
função de aptidão que mede o tempo necessário. Como o objectivo é a minimização, quanto menor o tempo correspondente a esse indivíduo mais apto ele é considerado. 

\cleardoublepage
\section{Implementação}
\indent \indent ...

\cleardoublepage

\subsection{...}
\indent \indent ...

\subsection{...}
\indent \indent ...

\subsubsection{...}
\indent \indent ...

\subsubsection{...}
\indent \indent ...

...

\cleardoublepage
\subsection{...}
\indent \indent ...

...

\subsection{...}
\indent \indent ...

...

\subsection{...}
\indent \indent ...

\cleardoublepage
\section{Validação}
\indent \indent ...

\cleardoublepage

\subsection{...}
\indent \indent ...

\cleardoublepage
\section{Experimentação}
\indent \indent ...

\cleardoublepage

\subsection{...}
\indent \indent ...

\cleardoublepage
\section{Análise de resultados}
\indent \indent ...

\cleardoublepage

\subsection{...}
\indent \indent ...

\cleardoublepage
\section{Conclusão}
\indent \indent ...

\cleardoublepage

\subsection{...}
\indent \indent ...


\end{document}
