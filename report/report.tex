\documentclass[a4paper]{article}

\usepackage[portuguese]{babel}
\usepackage[utf8]{inputenc}
\usepackage[T1]{fontenc}

\newcommand{\documentTitle}{Braitenberg Vehicles} %Macro definition
\newcommand{\documentAuthors}{João Rafael (2008111876, jprafael@student.dei.uc.pt) \and José Ribeiro (2008112181, jbaia@student.dei.uc.pt)} %Macro definition

\title{\documentTitle}
\author{\documentAuthors{}}

\usepackage{hyperref}
\hypersetup{
	pdftitle = \documentTitle
	,pdfauthor = \documentAuthors
	,pdfsubject = {Introduction to Artificial Inteligence Project \#1 Report}
	,pdfkeywords = {Artificial Inteligence Project} {Reactive Agents} {Braitenberg Vehicles}
	,pdfborder = {0 0 0}
}

\usepackage{subfig}
\usepackage{amsmath}
\usepackage{wrapfig}
\usepackage{array}
\usepackage{anysize}
\usepackage{lscape}
\usepackage[pdftex]{graphicx}

\marginsize{3.5cm}{3.5cm}{3cm}{3cm}

\makeatletter

\begin{document}
\renewcommand{\figurename}{Figure}
\maketitle
\cleardoublepage

\tableofcontents
\cleardoublepage

\setlength{\parindent}{1cm}
\setlength{\parskip}{0.3cm}

\section{Introduction}
% TODO

\cleardoublepage
\section{Breve Libraries}
\indent \indent Como referenciado na documentação da biblioteca Breve, o código Python fornecido é obtido através da compilação de código Steve.
Assim, esta biblioteca está pouco optimizada na medida em que não utiliza todas as potencialidades da linguagem.
Como tal, foram efectuadas algumas alterações no sentido de promover uma maior modularidade do código e flexibilidade.

\cleardoublepage

\subsection{Constructors with parameters}
\indent \indent Uma das funcionalidades não utilizadas pelo Breve é a construção de objectos com parâmetros.
Desta forma, para criar um objecto é necessária uma chamada extra a uma função (\texttt{init}).
Isto aumenta desnecessariamente o tamanho do código e dá azo a erros (utilização de objectos não inicializados).
Uma vez que todas as instâncias de objectos têm que ser criadas através da função \texttt{createInstances} decidimos alterá-la de forma a transmitir os parâmetros utilizados durante a sua criação.
Consequentemente, é possível especificá-los directamente aquando da construção de veículos e cenários.

\subsection{Object distance}
\indent \indent Para implementar correctamente os sensores, é necessário calcular a distância entre dois objectos.
Para o cálculo desta distância as bibliotecas originais apenas têm em conta a distância euclidiana entre os centros.
No entanto esta aproximação não é suficiente quando os objectos têm dimensões elevadas.

\subsubsection{Point - Sphere}
\indent \indent Por definição, todos os pontos da superficie esférica estão à mesma distância do centro.
Desta forma a solução para o caso das esferas é apenas considerar a distância entre os centros e subtrair o raio da esfera.

\subsubsection{Point - Box}
\indent \indent Uma solução eficiente para este caso consiste em utilizar o algoritmo de Arvo como \\ descrito em  \footnote[1]{\url{http://www.gamasutra.com/view/feature/3383/simple_intersection_tests_for_games.php?page=4}}.
Este algoritmo necessita que a Box esteja alinhada com os eixos. Como este não é originalmente o caso, é necessário transformar as coordenadas do ponto no referencial original $O$ para o referencial da box $B$.
Esta transformação é obtida através do seguinte produto de matrizes:

\[
 	\begin{bmatrix}
		P_{x}' \\
		P_{y}' \\
		P_{z}' \\
		1 
	\end{bmatrix}
	=
	\begin{bmatrix}
		x_{x} & x_{y} & x_{z} & \vline & O_{x}	\\
		y_{x} & y_{y} & y_{z} & \vline & O_{y}	\\
		z_{x} & z_{y} & z_{z} & \vline & O_{z}	\\
		0 & 0 & 0 & \vline & 1 	\\
	\end{bmatrix}
	*
 	\begin{bmatrix}
		P_{x} \\
		P_{y} \\
		P_{z} \\
		1 
	\end{bmatrix}
\]

Onde $x, y, z$ são os versores do referencial $B$ em relação ao referencial $O$ e $O_{x}, O_{y}, O_{z}$ são as coordenadas da origem do referencial $O$ nas coordenadas do referencial $B$.

\cleardoublepage
\subsection{Activators}
\indent \indent Os veículos de Braitenberg, como definidos na literatura apenas permitem relacionar um sensor directamente com uma única roda.
Esta abordagem implica a replicação de sensores quando se pretende que estes tenham influências diferentes para cada roda.
Para evitar esta duplicação introduzimos o conceito de \texttt{Activator}. 

Um \texttt{Activator} é um bloco conceptual introduzido entre vários sensores e uma roda.
Este é responsável pelo cálculo das funções de activação de cada sensor e posterior agregação dos resultados, enviando depois o sinal correspondente para a roda. Desta forma, o veículo continua a ser um veículo de Braitenberg desde que a agregação dos resultados seja efectuada apenas com operações elementares (i.e: adição). 

\subsection{Sensor rotation and positioning}
\indent \indent Após a criação de um sensor este fica orientado para cima (\texttt{dir.UP = (0,1,0)}).
A colocação e rotação destes é realizada no momento de ligação ao veículo a que pertencem.
De forma a facilitar a programação e leitura de código esta função recebe como parâmetro a direcção pretendida e calcula a rotação necessária para a obter.

Num caso genérico, a transformação de um versor (direcção inicial: $\vec{i}$) noutro (direcção final $\vec{f}$) é efectuada através de uma rotação de ângulo $\alpha$ sobre um eixo $\vec{n}$ (normal).
Estes podem ser calculados através de
\[
	\vec{n} = \vec{i}\ast\vec{f} \quad \text{(produto externo)}
\]
\[	
	\alpha = \arccos{(\vec{i} \cdot \vec{f})} \quad \text{(produto interno)}
\]

Neste caso em particular, visto conheçermos o vector inicial podemos efectuar a expanção dos produtos e obtemos:
\[
	\vec{n} = (-\vec{i}_{z}, 0, \vec{i}_{x})
\]
\[	
	\alpha = \arccos{(\vec{i}_{y})}
\]

\subsection{Multibody collision handlers (Proxies, and Real's parents )}
\indent \indent A biblioteca do Breve apenas permite a detecção de colisões entre \texttt{Shapes}. 
No entanto durante a execução do nosso projecto deparámo-nos com a necessidade de detectar a colisão entre dois objectos \texttt{MultiBody}.
Para este efeito são colocados \emph{handlers} específicos para cada componente, que por sua vez efectuam \emph{event bubbling} para o objecto desejado.

\cleardoublepage
\section{Sensors}
\indent \indent Uma das partes que compõem um veículo de Braitenberg (e agentes reactivos na sua generalidade) são os sensores.
Estes comportam toda a entrada de dados fornecidos ao agente e podem apresentar diferentes complexidades.
Quanto mais realistas os sensores forem melhor é a qualidade máxima teórica dos agentes uma vez que estes possuem mais informação.
No entanto, mais complexidade necessita de mais poder computacional, tornando-se mais difícil de analisar (um fenómeno denominado por \emph{information overloading}).
Desta forma, é necessario efectuar \emph{trade-offs} de forma a encontrar um equilíbrio adequado.

%\cleardoublepage
\subsection{Laser}

\begin{figure}[h]
	\vspace{-20pt}
	\begin{center}
		\includegraphics[width=0.6\textwidth]{graphs/sensors/laser.png}
	\end{center}
	\vspace{-20pt}
	\caption{Laser: $\alpha=\frac{\pi}{20}$}
\end{figure}

\indent Este sensor devolve simplesmente a existência ou não de objectos no seu ângulo, tendo assim um \emph{output} binário de -1 ou 1.
Assim, o \emph{output} do sensor é:
\[
	I = \left\{
		\begin{array}{lr}
			1 & : \text{caso exista um objecto no ângulo}\, \alpha \\
			-1 & : \text{caso contrário}
		\end{array}
		\right.
\]

onde $\alpha$ é a abertura do sensor ($rad$).


\cleardoublepage
\subsection{Distance}

\begin{figure}[h]
	\vspace{-20pt}
	\begin{center}
		\includegraphics[width=0.6\textwidth]{graphs/sensors/distance.png}
	\end{center}
	\vspace{-20pt}
	\caption{Distance: $\alpha=\frac{\pi}{4}$}
\end{figure}

\indent Este sensor permite obter a mínima distância do sensor aos objectos dentro do seu ângulo de visão.
Assim, o \emph{output} do sensor é:
\[
	I = \min\{o_{d},\quad o \in objects \land o_{\alpha} < \alpha\}
\] 

onde $\alpha$ é a abertura do sensor ($rad$).

Este é o único sensor a dar valores no intervalo [0 .. $+\infty$].

\cleardoublepage
\subsection{Proximity}
\begin{figure}[h]
	\vspace{-20pt}
	\begin{center}
		\includegraphics[width=0.6\textwidth]{graphs/sensors/proximity.png}
	\end{center}
	\vspace{-20pt}
	\caption{Proximity: \emph{half-distance}$=50$, $\alpha=\frac{\pi}{4}$}
\end{figure}

\indent O resultado deste sensor é inversamente proporcional à distância ao objecto mais próximo dentro do seu ângulo de visão.
O nível de decrescimento desta função é definido por \emph{half-distance} que indica a distância à qual se obtém o valor 0.5.
Desta forma a saída do sensor é:

\[
	I = \frac{1}{\frac{d_{min}}{d}+1},\quad d_{min} = \min\{o_{d},\quad o \in objects \land o_{\alpha} < \alpha\}
\]

onde $o_{d}$ é a distância do sensor ao objecto; $\alpha$ é o ângulo de visão do sensor e $d$ é a \emph{half-distance}.
Note-se que a função está deslocada uma unidade de forma a garantir valores no intervalo [0..1].

\cleardoublepage
\subsection{Smell}
\begin{figure}[h]
	\vspace{-20pt}
	\begin{center}
		\includegraphics[width=0.6\textwidth]{graphs/sensors/smell.png}
	\end{center}
	\vspace{-20pt}
	\caption{Smell: \emph{half-distance}$=50$}
\end{figure}

\indent Este sensor indica a intensidade de cheiro captada. Esta intensidade é dada por um somatório das intensidades de todas as fontes de cheiro que privilegia as mais próximas do sensor. O factor de importância corresponde ao inverso do quadrado da distância do sensor a cada uma das fontes.
Assim, a intensidade deste sensor é:
\[
	I = \displaystyle\sum\limits_{s \in smell} \frac{s_{i}}{1 + (\frac{s_{d}}{d})^{2}}
\] 

onde $s_{i}$ e $s_{d}$ são a intensidade e a distância de cada fonte de cheiro, respectivamente; $d$ é a distância à qual uma fonte de cheiro com intensidade 1 situada em frente ao sensor produz uma saída no sensor de 0.5 (\emph{half-distance}).

Este sensor é omni-direcional, ao contrário de todos os outros sensores; como tal este não possui ângulo de abertura. 

\cleardoublepage
\subsection{Light}
\begin{figure}[h]
	\vspace{-20pt}
	\begin{center}
		\includegraphics[width=0.6\textwidth]{graphs/sensors/light.png}
	\end{center}
	\vspace{-20pt}
	\caption{Light: \emph{half-distance}$=50$, $\alpha=\frac{\pi}{2}$}
\end{figure}

\indent Este sensor permite obter a intensidade de luz captada. Tal como no sensor de cheiro, a intensidade transmitida
é comulativa e inversamente proporcional ao quadrado da distância. No entanto, este sensor é direcionado (i.e. fontes
directamente à frente do sensor influenciam mais que os existentes na periferia). Assim, a intensidade total do sensor é:
\[
	I = \displaystyle\sum\limits_{l \in lights} \frac{l_{i}}{1 + (\frac{l_{d}}{d})^{2}*cos(\frac{2\pi}{\alpha}*l_{\alpha})}
\]

onde $l_{i}$, $l_{d}$, $l_{\alpha}$ são a intensidade, a distância e o ângulo de visão para cada luz, respectivamente;
$\alpha$ é a abertura do sensor ($rad$);
e $d$ é a distância à qual uma luz com intensidade 1 situada em frente ao sensor produz uma saída no sensor de 0.5.

\cleardoublepage
\subsection{Sound}
\begin{figure}[h]
	\vspace{-20pt}
	\begin{center}
		\includegraphics[width=0.6\textwidth]{graphs/sensors/cardioid.png}
	\end{center}
	\vspace{-20pt}
	\caption{Sound: \emph{half-distance}$=50$}
\end{figure}

\indent Este sensor indica a intensidade de som captado e pretende simular o
comportamento do ouvido humano. Desta forma, fontes de som em frente ao sensor têm mais impacto,
ainda que fontes atrás também o influenciem. Este comportamento foi bastante estudado e é aproximado pela função \emph{cardióide} definida em $\theta \in [0..2\pi]$:
\[
	cardioid(\theta) = \frac{1 + cos(\theta)}{2}
\]

\indent Extendendo esta função para 3 dimensões obtemos uma fórmula fechada para a superficie cardioidal
que utilizamos para construir o sensor de som cuja intensidade é
\[
	I = \displaystyle\sum\limits_{s \in sounds}\frac{1+cos(s_{\alpha})}{2}*\frac{s_{i}}{\frac{s_{d}}{d}+1}
\] 

onde $s_{i}$, $s_{d}$, $s_{\alpha}$ são a intensidade, a distância e o ângulo de fonte de som;
e $d$ a distância à qual uma uma fonte com intensidade 1 situada em frente ao sensor produz uma saída no sensor de 0.5.
Note-se que o resultado deste sensor é semelhante ao do sensor de luz quando $\alpha=\pi$,
mas decresce mais rapidamente quando o ângulo aumenta.

\cleardoublepage
\section{Vehicles}
%TODO

\subsection{Eight}
\begin{figure}[h]
	\centering

	\subfloat[Trail]{\includegraphics[width=0.2\textwidth]{trail/eight.png}}
	\subfloat[Left activator]{\includegraphics[width=0.4\textwidth]{graphs/activators/eight_l.png}}
	\subfloat[Right activator]{\includegraphics[width=0.4\textwidth]{graphs/activators/eight_r.png}}
	
	\caption{The eight vehicle}
\end{figure}
\indent O objectivo deste veículo é desenhar oitos à volta de duas esferas.

\indent Para este efeito recorremos a dois Proximity Sensors orientados para os lados (\texttt{dir.LEFT} e \texttt{dir.RIGHT}).
Desta forma é possivel definir 3 regiões onde os comportamentos são diferentes: apenas vê a esfera direita e deve rodar sobre esta; vê ambas as esferas e deve seguir em frente; apenas vê a esfera da esquerda e deve rodar sobre esta.
Assim, os coeficientes das funções de activação correspondem às velocidades tangenciais em circunferências de raio $r$ e $r+$\emph{axis\_dist}
\[
	|\vec{v_{tan}}| = r * \omega 
\]
onde $r$ é o raio; e $\omega$ é a velocidade ângular.

\indent Este veículo verifica o comportamento idealizado.
No entanto apresenta algumas oscilações devido a erros de precisão (distância entre as esferas, raio das esferas, posição inicial, etc.).
Apesar disto o seu comportamento é estável e mantém-se durante toda a simulação.

\cleardoublepage
\subsection{Ellipse}
\begin{figure}[h]
	\centering
	\subfloat[Trail]{\includegraphics[width=0.2\textwidth]{trail/ellipse.png}}
	\subfloat[Left activator]{\includegraphics[width=0.4\textwidth]{graphs/activators/ellipse_l.png}}
	\subfloat[Right activator]{\includegraphics[width=0.4\textwidth]{graphs/activators/ellipse_r.png}}
	
	\caption{The ellipse vehicle}
\end{figure}

\indent O objectivo deste veículo é desenhar uma elipse entre duas esferas.

Para este efeito recorremos a um Distance Sensor ($\alpha = \pi$) e a um Laser Sensor ($\alpha = \frac{\pi}{20}$).
Para $RADIUS$ igual ao raio da circunferência a descrever na proximidade da esfera, $DIFFERENTIAL$ igual à velocidade tangencial para descrever uma circunferência de raio $RADIUS$ (quando somado à natural) e $OFFSET$ a constante do valor a somar à velocidade natural para converger a sua rota para a esfera mais distante; são possíveis de definir três regiões onde os comportamentos diferem: \textbf{(1)} quando o veículo se encontra a uma distância menor ou igual a $RADIUS * 2$ de uma esfera (onde deverá descrever aproximadamente meia circunferência para sensivelmente inverter o sentido da marcha); \textbf{(2)} quando se encontra a uma distância superior a $RADIUS * 2$ e o Laser Sensor não se encontra a "ver" a esfera (pelo que deverá rodar $OFFSET$ até a ver); \textbf{(3)} quando se encontra a uma distância superior a $RADIUS * 2$ e o Laser Sensor "vê" a esfera (pelo que deverá continuar em linha recta). Assim, os coeficientes das funções de activação são (para a roda esquerda e direita, respectivamente), $NATURAL$, \textbf{(1)} $NATURAL + DIFFERENTIAL$, \textbf{(2)} $NATURAL + OFFSET$ e \textbf{(3)} $NATURAL$.

Este veículo verifica o comportamento idealizado. Descreve aproximadamente uma oval entre as esferas, ao curvar o suficiente para inverter a marcha e, gradualmente, converger para a segunda esfera; a partir do momento que a convergência é suficiente Laser Sensor corta essa convergência, de forma a garantir que o veículo não sai da rota esperada. Apesar de pequenas variações (derivadas da posição inicial e erros de precisão) o seu comportamento é estável e robusto; aquando de um desvio não esperado o veículo retoma a sua rota normal corrigindo o seu progresso (como demonstra a rota da figura). Pode-se mesmo variar as posições relativas das esferas (e até mesmo as distâncias) com alguma intensidade ao veículo usualmente responde correctamente.

\cleardoublepage
\subsection{Aggressor/Explorer}
\begin{figure}[h]
	\centering
	\subfloat[Left activator]{\includegraphics[width=0.4\textwidth]{graphs/activators/aggr_expl_l.png}}
	\subfloat[Right activator]{\includegraphics[width=0.4\textwidth]{graphs/activators/aggr_expl_r.png}}
	
	\caption{The Aggressor/Explorer vehicle}
\end{figure}

Este veículo vive num mundo com dois tipos de objectos: luzes e obstáculos.
O seu objectivo é explorá-lo, desviando-se dos obstáculos e explorando as luzes, i.e.,
aproxima-se delas como se estivesse curioso, após o qual perde o interesse seguindo à procura de novas.

Para se observar este comportamento a função de activação em relação aos obstáculos é decrescente até 0.5 (momento em que atinge \emph{half-distance}).
Após este valor assume-se que o veículo está preso e precisa de voltar para trás (rodar), pelo que se aplica uma intensidade simétrica em cada roda.

Quanto às luzes a função de activação cresce até 0.25 (o veículo vê uma luz e tenta aproximar-se rapidamente) e decresce até 0.5 (abranda para poder observar).
Após este valor a função assume valores cada vez menores (e negativos), de forma a afastar-se (uma vez que este sensor se encontra ligado de forma cruzada).

\cleardoublepage
\subsection{Braitenberg 3c}
\begin{figure}[h]
	\centering
	\subfloat[Left activator]{\includegraphics[width=0.5\textwidth]{graphs/activators/3c_l.png}}
	\subfloat[Right activator]{\includegraphics[width=0.5\textwidth]{graphs/activators/3c_r.png}}
	
	\caption{The Braitenberg 3c vehicle}
\end{figure}
Apesar do veículo sugerido possuir na mesma 4 sensores, este não é o Braitenberg 3c tal como definido por Valentino na sua obra.
No entanto, da mesma forma que Braitenberg avaliou a psicologia sintética dos seus veículos (regidos apenas por reações),
podemos avaliá-lo sob a mesma perspectiva.

Este veículo apresenta um comportamento bastante complexo. Pretende explorar todo o ambiente (luzes e obstáculos), gosta de 
correr no meio de flores (fontes de cheiro) e tem medo dos seus predadores (fontes de som). 

Para garantir este comportamento foram utilizadas funções de corte inibidoras quando as intensidades do sensores são pequenas.
Este corte evita que a soma dos vários comportamentos se acumule e o veículo se comporte de forma errática.
No entanto, quando 2 ou mais sensores apresentam intensidades superiores ao corte o comportamento corresponde à sua sobreposição.
Neste caso o resultado final pode não ser o melhor.


\cleardoublepage
\section{Project - Braitenberg Population}
\begin{figure}[h]
	\centering
	\subfloat[Group behaviour]{\includegraphics[width=0.3\textwidth]{screenshots/gossiping.png}}
	\hspace{5px}
	\subfloat[Sexual Opportunism]{\includegraphics[width=0.3\textwidth]{screenshots/opportunism.png}}
	\hspace{5px}
	\subfloat[Protective instinct]{\includegraphics[width=0.3\textwidth]{screenshots/protective.png}}

	\caption{Braitenberg Population}
\end{figure}

As populações de dadas espécies caracterizam-se por indivíduos com comportamentos distintos dentro da própria sociedade da qual fazem parte.
Esta experiência permite simular o comportamento de uma espécie fictícia (denominada Valentinum), com o objectivo de distinguir o papel dos indivíduos masculinos e femininos recorrendo apenas a simples reacções, com o objectivo de obter um comportamento emergente elaborado.

\subsection{Behaviours}

\subsubsection{Female}

\indent \indent As fêmeas da espécie Valentinum, ao contrário dos machos Valentinum, exibem comportamentos de grupo que podem ser vistos como parte da sua aptidão social. Estas comunicam através de sons entre si, sendo também atraídas pelo choro de crias, onde manifestam o seu instinto maternal e protector (dada a ausência de machos junto destas).

Assim, as fêmeas agrupam-se por forma a combater eventuais agressões externas e são elas o elo de ligação entre indivíduos Valentin\ae, podendo até dizer-se que estas criam o conceito de comunidade dentro desta espécie.

\subsubsection{Male}

\indent \indent Os machos da espécie Valentinum exibem um comportamento individual bastante demarcado do comportamento em grupo das fêmeas.

Enquanto jovens mostram-se bastante agressivos perante outros machos, tentando afastar-se das fêmeas.

Após atingirem a maturidade sexual, estes passam a evitar tensão com outros machos, concentrando o seu comportamento na reprodução e protecção da espécie. Usando o seu olfacto procuram parceiras sexuais, às quais se dirigem com maior intensidade consoante a sua proximidade. É atraído essencialmente para grandes grupos de fêmeas, onde a concentração de hormonas femininas é maior.

Após procriação, o macho tende a proteger o ovo até ao seu eclodir; fá-lo para garantir a sobrevivência das suas crias. No entanto, apesar desse instinto protector ser mais acentuado que o das fêmeas, estes não suportam o barulho, emitido (para além das fêmeas) pelos recém-nascidos. Como tal, após a eclosão, estes abandonam as crias, que serão mais tarde protegidas pelas fêmeas até à idade adulta.


\subsection{Implementation}
\subsubsection{Architecture}
\begin{tabular}{| c | c | c | c | c | c |}
 \hline
  \textbf{Gender} & \textbf{Sensor} & \textbf{Connection} & \textbf{Bias} & \textbf{Behaviour} & \textbf{Explanation} \\
 \hline
  & Light & Uncrossed & $-$ & Lover & Cares for its eggs \\
  Male & Proximity & Crossed & $-$ & Explorer &  \\
  & Smell & Crossed & $+$ & Aggressor & Tries to procriate \\
  & Sound & Uncrossed & $+$ & Coward & Runs away from babies \\
 \hline
  \textbf{Gender} & \textbf{Sensor} & \textbf{Connection} & \textbf{Bias} & \textbf{Behaviour} & \textbf{Explanation} \\
 \hline
  & Light & Uncrossed & $-$ & Lover & Cares for its eggs \\
  Female & Proximity & Crossed & $-$ & Explorer &  \\
  & Sound & Uncrossed & $-$ & Lover & Cares for its babies \\
 \hline
\end{tabular}

\subsubsection{Sources and Sensors}
\indent \indent Todos os veículos da espécie Valentin\ae apresentam uma estrutura semelhante:
\begin{itemize}
\item São quadrados;
\item Têm os sensores de proximidade e luz orientados para a frente;
\item Têm os sensores de som na diagonal, de forma a não ouvirem a eles próprios;
\item Têm uma fonte de som no meio cuja intensidade depende do sexo e da maturidade;
\item Têm uma fonte hormonal cuja intensidade depende da sua sexualidade.
\item Os machos têm sensores de cheiro orientados também para a frente.  
\end{itemize}

\cleardoublepage
\subsubsection{Activation functions}
\paragraph{Female}
\begin{figure}[ht]
	\centering
	\subfloat[Left activator]{\includegraphics[width=0.5\textwidth]{graphs/activators/female_l.png}}
	\subfloat[Right activator]{\includegraphics[width=0.5\textwidth]{graphs/activators/female_r.png}}
	\caption{The female Valentinum}
\end{figure}

\indent Este veículo possui uma velocidade natural de 2*\texttt{VELOCITY\_FACTOR} $m/s$.
Uma vez que este factor influência de igual forma todos os sentidos pode ser desprezado na análise das funções de activação. 
Assim quando quando a soma das funções atinge o simétrico deste valor o veículo pára, podendo mesmo recuar quando atinge valores inferiores.

\indent Para evitar o efeito cumulativo dos sensores, todas as funções possuem um corte a 0.2 que define uma distância mínima para a influência dos sensores.
O declive de cada função determina o ponto em que atinge o valor -2 e consequentemente a distância a que se tenta manter. 

\cleardoublepage
\paragraph{Male}
\begin{figure}[ht]
	\centering
	\subfloat[Left activator]{\includegraphics[width=0.3\textwidth]{graphs/activators/male_l.png}}
	\subfloat[Right activator]{\includegraphics[width=0.3\textwidth]{graphs/activators/male_r.png}}
	\subfloat[Smell function]{\includegraphics[width=0.3\textwidth]{graphs/activators/male_smell.png}}
	\caption{The male Valentinum}
\end{figure}

\indent A análise das funções de activação para a luz e para a distância procede-se da mesma forma que para a fêmea.
No entanto ao contrário desta, as restantes funções de activação do macho Valentinum não procedem deste modo.

\indent Uma vez que se pretende que o macho fuja de fontes de som, a função de activação cresce de forma linear com a
intensidade (uma vez que esta é inversamente porporcional ao quadrado da distância).

\indent A análise do comportamento relativo ao som é mais complexa. Cada \emph{activator} tem acesso a ambos os sensores de 
som. Desta forma é capaz de calcular tanto a soma como a diferença, valores que utiliza numa função de duas dimensões.
Esta função aplica-se apenas caso o macho esteja sexualmente activo e tem dois objectivos.
\begin{enumerate}
\item
Direcionar o veículo para a fêmea mais próxima. Para tal depende apenas da diferença dos sensores.
\item
Permitir ao macho tolerar o som emitido pela fêmea durante o acto da procriação.
Este feito é obtido através da aplicação de uma velocidade com intensidade suficiente quanto o veículo encontra-se simultâneamente a uma pequena distância e orientado para a fêmea.
\end{enumerate}

\subsubsection{Limitations}
\indent \indent Por causa do efeito de \emph{information overloading} já referido anteriormente, torna-se complexo codificar de forma exacta vários instintos. Mais difícil que codificá-los, torna-se complexo fazer o veículo, de forma puramente reactiva, obedecer a todos eles correctamente podendo, tendencialmente, falhar vários instintos por essa mesma tentativa de os satisfazer a todos eles.

Um exemplo disso pode verificar-se na seguinte situação: um macho perfeitamente alinhado com uma fêmea localizada atrás dele, estando este virado para uma parede. Dado o alinhamento, o diferencial entre os sensores de hormonas será mínimo (nulo no caso de alinhamento perfeito). Como tal, este agirá no sentido de fazer uma investida na fêmea, ainda que esta se encontre por trás de si. Isto levará a que este indivíduo choque contra a parede. Este comportamento poderia ser corrigido com a colocação de outros sensores que permitissem a detecção do sentido do alinhamento (positivo ou negativo), com funções de activação mais complexa que fizessem uso de múltiplos sensores, com a utilização dos sensores de visão em conjunto com os de hormonas, entre outras soluções.

\end{document}
