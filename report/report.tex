\documentclass[a4paper]{article}

\usepackage[portuguese]{babel}
\usepackage[utf8]{inputenc}
\usepackage[T1]{fontenc}

\newcommand{\documentTitle}{Braitenberg Vehivles} %Macro definition
\newcommand{\documentAuthors}{João Rafael (2008111876) \and José Ribeiro (2008112181)} %Macro definition

\title{\documentTitle}
\author{\documentAuthors{}}

\usepackage{hyperref}
\hypersetup{
	pdftitle = \documentTitle
	,pdfauthor = \documentAuthors
	,pdfsubject = {Introduction to Artificial Inteligence Project \#1 Report}
	,pdfkeywords = {Artificial Inteligence Project} {Reactive Agents} {Braitenberg Vehicles}
	,pdfborder = {0 0 0}
}

\usepackage{subfig}
\usepackage{amsmath}
\usepackage{wrapfig}
\usepackage{array}
\usepackage{anysize}
\usepackage{lscape}
\usepackage[pdftex]{graphicx}

\marginsize{3.5cm}{3.5cm}{3cm}{3cm}

\makeatletter

\begin{document}
\maketitle
\cleardoublepage

\tableofcontents
\cleardoublepage

\setlength{\parindent}{1cm}
\setlength{\parskip}{0.3cm}

\section{Introduction}
% TODO

\cleardoublepage
\section{Breve Libraries}
% TODO
\indent \indent Como referenciado na documentação da biblioteca Breve, o código Python fornecido é obtido através da compilação de código Steve.
Assim, esta biblioteca está pouco optimizada na medida em que não utiliza todas as potencialidades da linguagem. 
Por isso decidimos efectuar algumas alterações.

\subsection{Constructors with parameters}
\indent \indent Uma das funcionalidades não utilizadas pelo Breve é a construção de objectos com parâmetros.
Desta forma para criar um objecto é necessária uma chamada extra a uma função \emph{init}.
Isto aumenta desnecessáriamente o tamanho do código e dá azo a erros (utilização de objectos não inicializados).
Uma vez que todas as instâncias de objectos têm que ser criadas através da função \emph{createInstances}, decidimos altera-la de forma transmitir os parâmetros utilizados para a criação dos objectos. 
Consequentemente é possível alterar directamente os utilizados na construção de v																																																							eículos e cenários.

\subsection{Object distance}
\indent \indent Para implementar correctamente os sensores, é necessário calcular a distância entre dois objectos.
Para o cálculo desta distância as bibliotecas originais apenas têm em conta a distância euclidiana entre os centros.
No entanto esta aproximação não é suficiente quando os objectos têm dimensões elevadas.

\subsubsection{Point - Sphere}
\indent \indent Por definição todos os pontos da superficie esférica estão à mesma distância do centro.
Desta forma a solução para o caso das esferas é apenas considerar a distância entre os centros e subtrair o raio da esfera.

\subsubsection{Point - Box}
\indent \indent Uma solução eficiente para este caso consiste em utilizar o algoritmo de Arvo como \\ descrito em  \footnote[1]{\url{http://www.gamasutra.com/view/feature/3383/simple_intersection_tests_for_games.php?page=4}}.
No entanto este algoritmo necessita que a Box esteja alinhada com os eixos.
Como este não é originalmente o caso é necessário transformar as coordenadas do ponto no referencial original $O$ para o referencial da box $B$.
Esta transformação é obtida através do produto de matrizes:

\[
 	\begin{bmatrix}
		P_{x}' \\
		P_{y}' \\
		P_{z}' \\
		1 
	\end{bmatrix}
	=
	\begin{bmatrix}
		x_{x} & x_{y} & x_{z} & \vline & O_{x}	\\
		y_{x} & y_{y} & y_{z} & \vline & O_{y}	\\
		z_{x} & z_{y} & z_{z} & \vline & O_{z}	\\
		0 & 0 & 0 & \vline & 1 	\\
	\end{bmatrix}
	*
 	\begin{bmatrix}
		P_{x} \\
		P_{y} \\
		P_{z} \\
		1 
	\end{bmatrix}
\]

Onde $x, y, z$ são os versores do referencial $B$ em relação ao referencial $O$ e $O_{x}, O_{y}, O_{z}$ são as coordenadas da origem do rerefencial O nas coordenadas do referencial B. 

\cleardoublepage
\subsection{Activators}
\indent \indent Os veículos de Braitenberg como definidos na literatura apenas permitem relacionar um sensor directamente com uma única roda.
Esta abordagem implica a replicação de sensores quando se pretende que estes tenham influências diferentes para cada roda.
Para evitar esta duplicação introduzimos o conceito de \emph{Activador}. 

Um \emph{Activador} é um bloco conceptual introduzido entre vários sensores e uma roda.
Este é responsável pelo cálculo das funções de activação de cada sensor e posterior agregação dos resultados, enviando depois o sinal correspondente para a roda.  
Desta forma, o veículo continua a ser um veículo de Braitenberg desde que a agregação dos resultados seja efectuada com apenas operações elementares (i.e: adição). 

%TODO

\subsection{Sensor rotation and initialization}
\indent \indent 

\subsection{Multibody collision handlers (Proxies, and Real's parents )}
\indent \indent

\cleardoublepage
\section{Sensors}
\indent \indent Uma das partes que compõem um veículo de Braitenberg (e agentes reactivos na sua generalidade) são os sensores.
Estes comportam toda a entrada de dados fornecidos ao agente e podem apresentar diferentes complexidades.
Quanto mais realistas os sensores forem melhor é a qualidade máxima teorica dos agentes uma vez que estes possuem mais informação.
No entanto, mais complexidade necessita de mais poder computacional e e torna-se mais difícil analisar (um fenómeno denominado por \emph{information overloading}).
Desta forma é necessario efectuar \emph{trade-offs} de forma a encontrar um equilibrio adequado.

\cleardoublepage
\subsection{Laser}

\begin{figure}[h]
	\vspace{-20pt}
	\begin{center}
		\includegraphics[width=0.6\textwidth]{graphs/sensors/laser.png}
	\end{center}
	\vspace{-20pt}
	\caption{Laser: $\alpha=\frac{\pi}{20}$}
\end{figure}

TODO

\cleardoublepage
\subsection{Distance}

\begin{figure}[h]
	\vspace{-20pt}
	\begin{center}
		\includegraphics[width=0.6\textwidth]{graphs/sensors/distance.png}
	\end{center}
	\vspace{-20pt}
	\caption{Distance: $\alpha=\frac{\pi}{4}$}
\end{figure}

TODO

\cleardoublepage
\subsection{Proximity}

\begin{figure}[h]
	\vspace{-20pt}
	\begin{center}
		\includegraphics[width=0.6\textwidth]{graphs/sensors/proximity.png}
	\end{center}
	\vspace{-20pt}
	\caption{Proximity: $bias=50$ $\alpha=\frac{\pi}{4}$}
\end{figure}

TODO

\cleardoublepage
\subsection{Smell}
\begin{figure}[h]
	\vspace{-20pt}
	\begin{center}
		\includegraphics[width=0.6\textwidth]{graphs/sensors/smell.png}
	\end{center}
	\vspace{-20pt}
	\caption{Smell: $bias=50$}
\end{figure}

TODO

\cleardoublepage
\subsection{Light}
\begin{figure}[h]
	\vspace{-20pt}
	\begin{center}
		\includegraphics[width=0.6\textwidth]{graphs/sensors/light.png}
	\end{center}
	\vspace{-20pt}
	\caption{Light: $bias=50$ $\alpha=\frac{\pi}{2}$}
\end{figure}

\indent Este sensor permite obter a intensidade de luz captada. Tal como no sensor de cheiro a intensidade transmitida
é comulativa e inversamente proporsional ao quadrado da distância. No entanto este sensor é direcionado (i.e. fontes
directamente á frente do sensor influenciam mais que os existentes na periferia. Assim a intensidade total do sensor é:
\[
	I = \displaystyle\sum\limits_{l \in lights} \frac{l_{i}}{1 + (\frac{l_{d}}{d})^{2}*cos(\frac{2\pi}{\alpha}*l_{\alpha})}
\]

onde $l_{i}$, $l_{d}$, $l_{\alpha}$ são a intensidade, a distância e o ângulo de visão para cada luz;
$\alpha$ é a abertura do sensor ($rad$);
e $d$ é a distância á qual uma luz com intensidade 1 situada em frente ao sensor produz uma saida no sensor de 0.5.

\cleardoublepage
\subsection{Sound}
\begin{figure}[h]
	\vspace{-20pt}
	\begin{center}
		\includegraphics[width=0.6\textwidth]{graphs/sensors/cardioid.png}
	\end{center}
	\vspace{-20pt}
	\caption{Sound: $bias=50$}
\end{figure}

\indent Este sensor indica a intensidade de som captado e pretende simular o
comportamento do ouvido humano. Desta forma fontes de som em frente ao sensor têm mais impacto
mas fontes atrás deste também o influenciam. Este comportamento foi bastante estudado e é aproximado pela função \emph{cardioide} definida em $\theta \in [0..2\pi]$:
\[
	cardioid(\theta) = \frac{1 + cos(\theta)}{2}
\]

\indent Extendendo esta função para 3 dimensões obtemos uma fórmula fechada para a superficie cardioidal
que utilizamos para construir o sensor de som cuja intensidade é
\[
	I = \displaystyle\sum\limits_{s \in sounds}\frac{1+cos(s_{\alpha})}{2}*\frac{s_{i}}{\frac{s_{d}}{d}+1}
\] 

onde $s_{i}$, $s_{d}$, $s_{\alpha}$ são a intensidade, a distância e o ângulo de fonte de som;
e $d$ é a distância à qual uma uma fonte com intensidade 1 situada em frente ao sensor produz uma saída no sensor de 0.5.
Note-se que o resultado deste sensor é semelhante ao do sensor de luz quando $\alpha=\pi$,
mas decresce mais rapidamente quando o ângulo aumenta.

\cleardoublepage
\section{Vehicles}
%TODO

\subsection{Eight}
\begin{figure}[h]
	\centering

	\subfloat[Trail]{\includegraphics[width=0.2\textwidth]{trail/eight.png}}
	\subfloat[Left activator]{\includegraphics[width=0.4\textwidth]{graphs/activators/eight_l.png}}
	\subfloat[Right activator]{\includegraphics[width=0.4\textwidth]{graphs/activators/eight_r.png}}
	
	\caption{The eight vehicle}
\end{figure}


\cleardoublepage
\subsection{Ellipse}
\begin{figure}[h]
	\centering
	\subfloat[Trail]{\includegraphics[width=0.2\textwidth]{trail/ellipse.png}}
	\subfloat[Left activator]{\includegraphics[width=0.4\textwidth]{graphs/activators/ellipse_l.png}}
	\subfloat[Right activator]{\includegraphics[width=0.4\textwidth]{graphs/activators/ellipse_r.png}}
	
	\caption{The ellipse vehicle}
\end{figure}

\cleardoublepage
\subsection{Aggressor Explorer}
\begin{figure}[h]
	\centering
	\subfloat[Left activator]{\includegraphics[width=0.4\textwidth]{graphs/activators/aggr_expl_l.png}}
	\subfloat[Right activator]{\includegraphics[width=0.4\textwidth]{graphs/activators/aggr_expl_r.png}}
	
	\caption{The Aggressor/Explorer vehicle}
\end{figure}

Este veículo vive num mundo com dois tipos de objectos: luzes e obstáculos.
O seu objectivo é explorar-lo, desviando-se dos obstáculos e explorando as luzes, i.e.
aproxima-se delas como se estivesse curioso e assim que perde o interesse segue à procura de novas.

Para se observar este comportamento a função de activação em relação aos obstáculos é decrescente até 0.5 (momento em que atinge \emph{HALF\_DISTANCE}).
Após este valor assume-se que o veículo está preso e precisa de voltar para trás (rodar) e por isso aplica-se uma intensidade simétrica em cada roda.

Quanto às luzes a função de activação cresce até 0.25 (o veículo vê uma luz e tenta se aproximar rapidamente) e decresce até 0.5 (abranda para poder observar).
Após este valor a função assume valores cada vez mais negativos de forma a se afastar (uma vez que este sensor se encontra ligado de forma cruzada).

\cleardoublepage
\subsection{Braitenberg 3c}
\begin{figure}[h]
	\centering
	\subfloat[Left activator]{\includegraphics[width=0.5\textwidth]{graphs/activators/3c_l.png}}
	\subfloat[Right activator]{\includegraphics[width=0.5\textwidth]{graphs/activators/3c_r.png}}
	
	\caption{The Braitenberg 3c vehicle}
\end{figure}
Apesar do veículo sugerido possuir na mesma 4 sensores este não é o Braitenberg 3c tal como definido na sua obra.
No entanto, da mesma forma que Valentino Braitenberg avaliou a psicologia sintética dos seus veículos (regidos apenas por reações),
podemos avalia-lo sob a mesma prespectiva:

Este veículo apresenta um comportamento bastante complexo. Pretende explorar todo o ambiente (luzes e obstáculos), gosta de 
correr no meio de flores (fontes de cheiro) e tem medo dos seus predadores (fontes de som). 

Para garantir este comportamento foram utilizadas funções de corte que inibidoras quando as intensidades do sensores são pequenas.
Este corte evita que a soma dos vários comportamentos se acumule e o veículo se comporte de forma errática.
No entanto, quando 2 ou mais sensores apresentam intensidades superiores ao corte o comportamento corresponde à sua sobreposição.
Neste caso o resultado final pode não ser o melhor.

% TODO Strenghts, weaknesses, reasons behind the functions, behaviour

\cleardoublepage
\section{Project}


\end{document}
